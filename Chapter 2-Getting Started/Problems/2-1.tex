\documentclass{article}
\usepackage{amssymb}
\usepackage{amsmath}
\usepackage{setspace}
\begin{document}
\begin{spacing}{1.5}
\noindent
\textbf{Exercise 2-1:}\\
a. Answer:\\
For an array of size k, sorting it using insertion sort will cost $\Theta(k^2)$ time.\\
Thus for n / k such sublists, the total running time is $\Theta(n / k \cdot k^2) = \Theta(nk)$\\

\noindent
b. Answer:\\
During the merging process, we pick the smallest element and put it in the merged list.\\
That's the "n" in $\Theta(n\lg (n / k))$\\
By using a minimal heap, we can find the smallest element in O(1) time.\\
But adjusting the heap requires $O(\lg (n / k))$ time.
Thus the total running time will be $\Theta(n\lg (n / k))$ in the worst case.\\

\noindent
c. Answer:\\
$
Let\ n\lg n = nk + n\lg (n / k)\\
\therefore n\lg n = nk + n\lg n - n\lg k\\
\therefore n\lg k = nk\\
\therefore k = \lg k,\ no\ solution.\\
\because But\ there\ is\ actually\ a\ constant\ c \ne 1,\ s.t.\ cn\lg n = nk + n\lg (n / k)\\
\therefore There\ could\ be\ a\ k\ that\ make\ them\ equal,\ but\ it\ will\ be\ fairly\ small.\\
$

\noindent
d. Answer:\\
When k=1, T(n)=$\Theta(n\lg n)$, it's degenerated to heap sort, which is good enough.\\
When k=n, T(n)=$\Theta(n^2)$, it's degenerated to insertion sort, which is good enough for small data sets.\\
So, make it small. How about 5?
\end{spacing}
\end{document}
