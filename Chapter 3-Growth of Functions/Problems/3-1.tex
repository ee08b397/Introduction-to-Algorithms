\documentclass{article}
\usepackage{amsfonts}
\usepackage{amsmath}
\usepackage{amssymb}
\usepackage{setspace}
\begin{document}
\begin{spacing}{2.0}
\noindent
\textbf{Exercise 3-1:}\\
$
a.\ Answer:\\
\because \forall c > 0, \exists n_0 \in N^*, s.t.\ \forall n > n_0, 0 \le p(n) \le cn^k\\
\therefore p(n) = O(n^k)\\
b.\ Answer:\\
\because \forall c > 0, \exists n_0 \in N^*, s.t.\ \forall n > n_0, 0 \le cn^k \le p(n)\\
\therefore p(n) = \Omega(n^k)\\
c.\ Answer:\\
\because \forall c_1, c_2 > 0, c1 < c2, \exists n_0 \in N^*, s.t.\ 
\forall n > n_0, c_1n^k \le p(n) \le c_2n^k\\
\therefore p(n) = \Theta(n^k)\\
d.\ Answer:\\
\because \forall c > 0, \exists n_0 \in N^*, s.t.\ \forall n > n_0, 0 \le p(n) < cn^k\\
\therefore p(n) = o(n^k)\\
e.\ Answer:\\
\because \forall c > 0, \exists n_0 \in N^*, s.t.\ \forall n > n_0, 0 \le cn^k < p(n)\\
\therefore p(n) = \omega(n^k)\\
$
\end{spacing}
\end{document}
